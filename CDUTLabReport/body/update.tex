\section{更新说明}
本文档为成都理工大学实验报告\LaTeX{}模板第一次版本改动,将模板设置内容封装成cdutlabreport文档类,适应TeXlive2019,同时更新了字体字号等设置,对模板一些设置做了更精细调整,方便用户使用。下面列出此次更新的内容:
\begin{itemize}
	\item 封装了全局配置为文档类;
	\item 设置了变量控制学生姓名等变量,不用进入cover文件更改原文档;
	\item 调整了封面页标题与下划线的间距,看起来更好看;
	\item 设置了日期区间,鉴于很多学生反应日期有起始和末了两个点;
	\item 对超链接添加了颜色,看起来更好看;
	\item 同时更新了超链接地址;
	\item 设置了目录不显示三级标题;
	\item 从titleformat全部更新为ctexset,文档类更改为ctexart,设置全部更新为ctex方法。
	\item 删除了致谢(因为没人写);
	\item 更改“学院名称”为“实验名称”;
	\item 更新字体为“宋体”,支持textbf加粗;
	\item 更新从封面到正文的字号;
	\item 更新名称从“CDUT\_Lab\_report”为“CDUTLabReport”;
	\item 自主编写了文档类之后,删除了导入包文档和设置文档;
	\item 单独设置了coverfig文件夹,存放“成都理工大学”字样的图片。。;
	\item 设置了自动添加\verb|figure/|,导入图片可以直接写图片名了;
	\item 同时支持部分图文件夹名,如“figure”、“picture”等;
	\item 设置了图片、表格、公式编号跟随一级标题编号。
\end{itemize}
