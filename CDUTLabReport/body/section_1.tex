\section{模板使用基本知识}
文档模板严格按照成都理工大学本科生实验报告模板撰写,部分略有不同。纠正了原word版本中一些错误,严格按照国际文档书写排版标准重新定制内容格式。本模板适用于\textbf{新手},讲述内容不全面也不够精细,但是对于新人书写实验报告足够了,想要精细学习的读者可以阅读\href{http://mirror.lzu.edu.cn/CTAN/info/lshort/chinese/lshort-zh-cn.pdf}{lshort}开始,进阶可以阅读刘海洋的书籍。对本模板有意见或者不懂之处欢迎致电作者,邮箱:549274614@qq.com,或加入本校交流群:77586170。

在开始介绍前首先介绍我们所接触的\LaTeX 中的3种名词,分别是套装发行版、编译引擎与编译器。
\begin{itemize}
\item \textbf{套装发行版:}俗称\LaTeX 的核心,常见的有CTeX套装、TeXlive套装(作者使用)、MacTeX套装等等,本文使用TeXlive2019版本,下载地址为\url{http://tug.org/texlive/}。

\item \textbf{编译引擎:}也是编译时点的按钮,本文采用魔法注释设置了编译引擎为\hologo{XeLaTeX} ,也是中文文档排版常用的编译引擎,其他的编译引擎有\hologo{pdfLaTeX}、\LaTeX{}、\hologo{BibTeX}等等。
 
\item \textbf{编辑器:}顾名思义,编辑器就是编辑\hologo{TeX}文档所用的软件,编写本文档使用的是TeXstudio,也是作者推荐的一款编辑器,网上可以免费下载。下载tl2018时也会自带编辑器,名为TeXwork,也是一款很好的编辑器,编辑器的种类很多,甚至于word都可以拿来写\hologo{TeX}代码,只不过\hologo{TeX}编辑器拥有自带的语法高亮、自动补全等快捷方式。
\end{itemize}

\textbf{阅读或使用本模板时建议与\hologo{TeX}文档对比学习。}
\subsection{文件结构}
解压压缩包后,内含的文件内容与结构如下
\begin{itemize}
\item \textbf{成都理工大学实验报告模板.pdf}:本文件,内包括写作方法与注意事项;
\item \textbf{CDUTLabReport-demo.tex}:实验报告主文件,包括报告整体结构与一些全局定义;
\item \textbf{attachment文件夹}:内含文件attachment.pdf与attachment.docx,为实验报告中的心得体会,具体书写方式详见文末心得体会处;
\item \textbf{body文件夹}:内含各章节\hologo{TeX}文档(如section\_1.tex),方便起见,每一章(每一实验)单独使用一个文档书写;
\item \textbf{figure文件夹}:文档中各类图片依照命名规范至于此文件夹中,方便统一管理,在导入图片时可以不用添加\verb|figure/|;
\item \textbf{cover文件夹}:内含文档cover.tex为首页封面与目录设置,\textbf{除特殊情况请勿更改};
\item \textbf{coverfig文件夹}:内含“成都理工大学”封面图片,\textbf{除特殊情况请勿更改};
\item \textbf{texclear.bat}:用于清理辅助文件。
\item \textbf{cdutlabreport.cls}:成都理工大学实验报告类文档,其中包括格式等全局设置,\textbf{除特殊情况请勿更改}
\end{itemize}
\subsection{文档结构}
打开文件CDUTLabReport.tex文件后,内为文档内容结构
\begin{minted}{LaTeX}
% !TeX encoding = UTF8
% !TEX TXS-program:compile = txs:///xelatex/[-shell-escape]
\documentclass{cdutlabreport}
%----------------------------------------title page setting-----------------
\course{\LaTeX 书写行为规范}
\academyname{利用\LaTeX 书写成都理工大学实验报告}
\profession{专业全称(有专业方向的用小括号标明)}
\studentName{您的姓名}
\studentId{您的学号}
\teacherName{您的授课或者指导老师}
\location{授课地点(如:6C403)}
\grade{由指导老师书写}
\startDate{二〇一九年十二月}
\lastDate{二〇二〇年二月}

\begin{document}
\maketitle
\newpage
\tableofcontents
\thispagestyle{empty}
\newpage
\pagestyle{plain}
\setcounter{page}{1}
%----------------------------------------body--------------------------------
\section{更新说明}
本文档为成都理工大学实验报告\LaTeX{}模板第一次版本改动,将模板设置内容封装成cdutlabreport文档类,适应TeXlive2019,同时更新了字体字号等设置,对模板一些设置做了更精细调整,方便用户使用。下面列出此次更新的内容:
\begin{itemize}
	\item 封装了全局配置为文档类;
	\item 设置了变量控制学生姓名等变量,不用进入cover文件更改原文档;
	\item 调整了封面页标题与下划线的间距,看起来更好看;
	\item 设置了日期区间,鉴于很多学生反应日期有起始和末了两个点;
	\item 对超链接添加了颜色,看起来更好看;
	\item 同时更新了超链接地址;
	\item 设置了目录不显示三级标题;
	\item 从titleformat全部更新为ctexset,文档类更改为ctexart,设置全部更新为ctex方法。
	\item 删除了致谢(因为没人写);
	\item 更改“学院名称”为“实验名称”;
	\item 更新字体为“宋体”,支持textbf加粗;
	\item 更新从封面到正文的字号;
	\item 更新名称从“CDUT\_Lab\_report”为“CDUTLabReport”;
	\item 自主编写了文档类之后,删除了导入包文档和设置文档;
	\item 单独设置了coverfig文件夹,存放“成都理工大学”字样的图片。。;
	\item 设置了自动添加\verb|figure/|,导入图片可以直接写图片名了;
	\item 同时支持部分图文件夹名,如“figure”、“picture”等;
	\item 设置了图片、表格、公式编号跟随一级标题编号。
\end{itemize}

\section{模板使用基本知识}
文档模板严格按照成都理工大学本科生实验报告模板撰写,部分略有不同。纠正了原word版本中一些错误,严格按照国际文档书写排版标准重新定制内容格式。本模板适用于\textbf{新手},讲述内容不全面也不够精细,但是对于新人书写实验报告足够了,想要精细学习的读者可以阅读\href{http://mirror.lzu.edu.cn/CTAN/info/lshort/chinese/lshort-zh-cn.pdf}{lshort}开始,进阶可以阅读刘海洋的书籍。对本模板有意见或者不懂之处欢迎致电作者,邮箱:549274614@qq.com,或加入本校交流群:77586170。

在开始介绍前首先介绍我们所接触的\LaTeX 中的3种名词,分别是套装发行版、编译引擎与编译器。
\begin{itemize}
\item \textbf{套装发行版:}俗称\LaTeX 的核心,常见的有CTeX套装、TeXlive套装(作者使用)、MacTeX套装等等,本文使用TeXlive2019版本,下载地址为\url{http://tug.org/texlive/}。

\item \textbf{编译引擎:}也是编译时点的按钮,本文采用魔法注释设置了编译引擎为\hologo{XeLaTeX} ,也是中文文档排版常用的编译引擎,其他的编译引擎有\hologo{pdfLaTeX}、\LaTeX{}、\hologo{BibTeX}等等。
 
\item \textbf{编辑器:}顾名思义,编辑器就是编辑\hologo{TeX}文档所用的软件,编写本文档使用的是TeXstudio,也是作者推荐的一款编辑器,网上可以免费下载。下载tl2018时也会自带编辑器,名为TeXwork,也是一款很好的编辑器,编辑器的种类很多,甚至于word都可以拿来写\hologo{TeX}代码,只不过\hologo{TeX}编辑器拥有自带的语法高亮、自动补全等快捷方式。
\end{itemize}

\textbf{阅读或使用本模板时建议与\hologo{TeX}文档对比学习。}
\subsection{文件结构}
解压压缩包后,内含的文件内容与结构如下
\begin{itemize}
\item \textbf{成都理工大学实验报告模板.pdf}:本文件,内包括写作方法与注意事项;
\item \textbf{CDUTLabReport-demo.tex}:实验报告主文件,包括报告整体结构与一些全局定义;
\item \textbf{attachment文件夹}:内含文件attachment.pdf与attachment.docx,为实验报告中的心得体会,具体书写方式详见文末心得体会处;
\item \textbf{body文件夹}:内含各章节\hologo{TeX}文档(如section\_1.tex),方便起见,每一章(每一实验)单独使用一个文档书写;
\item \textbf{figure文件夹}:文档中各类图片依照命名规范至于此文件夹中,方便统一管理,在导入图片时可以不用添加\verb|figure/|;
\item \textbf{cover文件夹}:内含文档cover.tex为首页封面与目录设置,\textbf{除特殊情况请勿更改};
\item \textbf{coverfig文件夹}:内含“成都理工大学”封面图片,\textbf{除特殊情况请勿更改};
\item \textbf{texclear.bat}:用于清理辅助文件。
\item \textbf{cdutlabreport.cls}:成都理工大学实验报告类文档,其中包括格式等全局设置,\textbf{除特殊情况请勿更改}
\end{itemize}
\subsection{文档结构}
打开文件CDUTLabReport.tex文件后,内为文档内容结构
\begin{minted}{LaTeX}
% !TeX encoding = UTF8
% !TEX TXS-program:compile = txs:///xelatex/[-shell-escape]
\documentclass{cdutlabreport}
%----------------------------------------title page setting-----------------
\course{\LaTeX 书写行为规范}
\academyname{利用\LaTeX 书写成都理工大学实验报告}
\profession{专业全称(有专业方向的用小括号标明)}
\studentName{您的姓名}
\studentId{您的学号}
\teacherName{您的授课或者指导老师}
\location{授课地点(如:6C403)}
\grade{由指导老师书写}
\startDate{二〇一九年十二月}
\lastDate{二〇二〇年二月}

\begin{document}
\maketitle
\newpage
\tableofcontents
\thispagestyle{empty}
\newpage
\pagestyle{plain}
\setcounter{page}{1}
%----------------------------------------body--------------------------------
\section{更新说明}
本文档为成都理工大学实验报告\LaTeX{}模板第一次版本改动,将模板设置内容封装成cdutlabreport文档类,适应TeXlive2019,同时更新了字体字号等设置,对模板一些设置做了更精细调整,方便用户使用。下面列出此次更新的内容:
\begin{itemize}
	\item 封装了全局配置为文档类;
	\item 设置了变量控制学生姓名等变量,不用进入cover文件更改原文档;
	\item 调整了封面页标题与下划线的间距,看起来更好看;
	\item 设置了日期区间,鉴于很多学生反应日期有起始和末了两个点;
	\item 对超链接添加了颜色,看起来更好看;
	\item 同时更新了超链接地址;
	\item 设置了目录不显示三级标题;
	\item 从titleformat全部更新为ctexset,文档类更改为ctexart,设置全部更新为ctex方法。
	\item 删除了致谢(因为没人写);
	\item 更改“学院名称”为“实验名称”;
	\item 更新字体为“宋体”,支持textbf加粗;
	\item 更新从封面到正文的字号;
	\item 更新名称从“CDUT\_Lab\_report”为“CDUTLabReport”;
	\item 自主编写了文档类之后,删除了导入包文档和设置文档;
	\item 单独设置了coverfig文件夹,存放“成都理工大学”字样的图片。。;
	\item 设置了自动添加\verb|figure/|,导入图片可以直接写图片名了;
	\item 同时支持部分图文件夹名,如“figure”、“picture”等;
	\item 设置了图片、表格、公式编号跟随一级标题编号。
\end{itemize}

\section{模板使用基本知识}
文档模板严格按照成都理工大学本科生实验报告模板撰写,部分略有不同。纠正了原word版本中一些错误,严格按照国际文档书写排版标准重新定制内容格式。本模板适用于\textbf{新手},讲述内容不全面也不够精细,但是对于新人书写实验报告足够了,想要精细学习的读者可以阅读\href{http://mirror.lzu.edu.cn/CTAN/info/lshort/chinese/lshort-zh-cn.pdf}{lshort}开始,进阶可以阅读刘海洋的书籍。对本模板有意见或者不懂之处欢迎致电作者,邮箱:549274614@qq.com,或加入本校交流群:77586170。

在开始介绍前首先介绍我们所接触的\LaTeX 中的3种名词,分别是套装发行版、编译引擎与编译器。
\begin{itemize}
\item \textbf{套装发行版:}俗称\LaTeX 的核心,常见的有CTeX套装、TeXlive套装(作者使用)、MacTeX套装等等,本文使用TeXlive2019版本,下载地址为\url{http://tug.org/texlive/}。

\item \textbf{编译引擎:}也是编译时点的按钮,本文采用魔法注释设置了编译引擎为\hologo{XeLaTeX} ,也是中文文档排版常用的编译引擎,其他的编译引擎有\hologo{pdfLaTeX}、\LaTeX{}、\hologo{BibTeX}等等。
 
\item \textbf{编辑器:}顾名思义,编辑器就是编辑\hologo{TeX}文档所用的软件,编写本文档使用的是TeXstudio,也是作者推荐的一款编辑器,网上可以免费下载。下载tl2018时也会自带编辑器,名为TeXwork,也是一款很好的编辑器,编辑器的种类很多,甚至于word都可以拿来写\hologo{TeX}代码,只不过\hologo{TeX}编辑器拥有自带的语法高亮、自动补全等快捷方式。
\end{itemize}

\textbf{阅读或使用本模板时建议与\hologo{TeX}文档对比学习。}
\subsection{文件结构}
解压压缩包后,内含的文件内容与结构如下
\begin{itemize}
\item \textbf{成都理工大学实验报告模板.pdf}:本文件,内包括写作方法与注意事项;
\item \textbf{CDUTLabReport-demo.tex}:实验报告主文件,包括报告整体结构与一些全局定义;
\item \textbf{attachment文件夹}:内含文件attachment.pdf与attachment.docx,为实验报告中的心得体会,具体书写方式详见文末心得体会处;
\item \textbf{body文件夹}:内含各章节\hologo{TeX}文档(如section\_1.tex),方便起见,每一章(每一实验)单独使用一个文档书写;
\item \textbf{figure文件夹}:文档中各类图片依照命名规范至于此文件夹中,方便统一管理,在导入图片时可以不用添加\verb|figure/|;
\item \textbf{cover文件夹}:内含文档cover.tex为首页封面与目录设置,\textbf{除特殊情况请勿更改};
\item \textbf{coverfig文件夹}:内含“成都理工大学”封面图片,\textbf{除特殊情况请勿更改};
\item \textbf{texclear.bat}:用于清理辅助文件。
\item \textbf{cdutlabreport.cls}:成都理工大学实验报告类文档,其中包括格式等全局设置,\textbf{除特殊情况请勿更改}
\end{itemize}
\subsection{文档结构}
打开文件CDUTLabReport.tex文件后,内为文档内容结构
\begin{minted}{LaTeX}
% !TeX encoding = UTF8
% !TEX TXS-program:compile = txs:///xelatex/[-shell-escape]
\documentclass{cdutlabreport}
%----------------------------------------title page setting-----------------
\course{\LaTeX 书写行为规范}
\academyname{利用\LaTeX 书写成都理工大学实验报告}
\profession{专业全称(有专业方向的用小括号标明)}
\studentName{您的姓名}
\studentId{您的学号}
\teacherName{您的授课或者指导老师}
\location{授课地点(如:6C403)}
\grade{由指导老师书写}
\startDate{二〇一九年十二月}
\lastDate{二〇二〇年二月}

\begin{document}
\maketitle
\newpage
\tableofcontents
\thispagestyle{empty}
\newpage
\pagestyle{plain}
\setcounter{page}{1}
%----------------------------------------body--------------------------------
\section{更新说明}
本文档为成都理工大学实验报告\LaTeX{}模板第一次版本改动,将模板设置内容封装成cdutlabreport文档类,适应TeXlive2019,同时更新了字体字号等设置,对模板一些设置做了更精细调整,方便用户使用。下面列出此次更新的内容:
\begin{itemize}
	\item 封装了全局配置为文档类;
	\item 设置了变量控制学生姓名等变量,不用进入cover文件更改原文档;
	\item 调整了封面页标题与下划线的间距,看起来更好看;
	\item 设置了日期区间,鉴于很多学生反应日期有起始和末了两个点;
	\item 对超链接添加了颜色,看起来更好看;
	\item 同时更新了超链接地址;
	\item 设置了目录不显示三级标题;
	\item 从titleformat全部更新为ctexset,文档类更改为ctexart,设置全部更新为ctex方法。
	\item 删除了致谢(因为没人写);
	\item 更改“学院名称”为“实验名称”;
	\item 更新字体为“宋体”,支持textbf加粗;
	\item 更新从封面到正文的字号;
	\item 更新名称从“CDUT\_Lab\_report”为“CDUTLabReport”;
	\item 自主编写了文档类之后,删除了导入包文档和设置文档;
	\item 单独设置了coverfig文件夹,存放“成都理工大学”字样的图片。。;
	\item 设置了自动添加\verb|figure/|,导入图片可以直接写图片名了;
	\item 同时支持部分图文件夹名,如“figure”、“picture”等;
	\item 设置了图片、表格、公式编号跟随一级标题编号。
\end{itemize}

\section{模板使用基本知识}
文档模板严格按照成都理工大学本科生实验报告模板撰写,部分略有不同。纠正了原word版本中一些错误,严格按照国际文档书写排版标准重新定制内容格式。本模板适用于\textbf{新手},讲述内容不全面也不够精细,但是对于新人书写实验报告足够了,想要精细学习的读者可以阅读\href{http://mirror.lzu.edu.cn/CTAN/info/lshort/chinese/lshort-zh-cn.pdf}{lshort}开始,进阶可以阅读刘海洋的书籍。对本模板有意见或者不懂之处欢迎致电作者,邮箱:549274614@qq.com,或加入本校交流群:77586170。

在开始介绍前首先介绍我们所接触的\LaTeX 中的3种名词,分别是套装发行版、编译引擎与编译器。
\begin{itemize}
\item \textbf{套装发行版:}俗称\LaTeX 的核心,常见的有CTeX套装、TeXlive套装(作者使用)、MacTeX套装等等,本文使用TeXlive2019版本,下载地址为\url{http://tug.org/texlive/}。

\item \textbf{编译引擎:}也是编译时点的按钮,本文采用魔法注释设置了编译引擎为\hologo{XeLaTeX} ,也是中文文档排版常用的编译引擎,其他的编译引擎有\hologo{pdfLaTeX}、\LaTeX{}、\hologo{BibTeX}等等。
 
\item \textbf{编辑器:}顾名思义,编辑器就是编辑\hologo{TeX}文档所用的软件,编写本文档使用的是TeXstudio,也是作者推荐的一款编辑器,网上可以免费下载。下载tl2018时也会自带编辑器,名为TeXwork,也是一款很好的编辑器,编辑器的种类很多,甚至于word都可以拿来写\hologo{TeX}代码,只不过\hologo{TeX}编辑器拥有自带的语法高亮、自动补全等快捷方式。
\end{itemize}

\textbf{阅读或使用本模板时建议与\hologo{TeX}文档对比学习。}
\subsection{文件结构}
解压压缩包后,内含的文件内容与结构如下
\begin{itemize}
\item \textbf{成都理工大学实验报告模板.pdf}:本文件,内包括写作方法与注意事项;
\item \textbf{CDUTLabReport-demo.tex}:实验报告主文件,包括报告整体结构与一些全局定义;
\item \textbf{attachment文件夹}:内含文件attachment.pdf与attachment.docx,为实验报告中的心得体会,具体书写方式详见文末心得体会处;
\item \textbf{body文件夹}:内含各章节\hologo{TeX}文档(如section\_1.tex),方便起见,每一章(每一实验)单独使用一个文档书写;
\item \textbf{figure文件夹}:文档中各类图片依照命名规范至于此文件夹中,方便统一管理,在导入图片时可以不用添加\verb|figure/|;
\item \textbf{cover文件夹}:内含文档cover.tex为首页封面与目录设置,\textbf{除特殊情况请勿更改};
\item \textbf{coverfig文件夹}:内含“成都理工大学”封面图片,\textbf{除特殊情况请勿更改};
\item \textbf{texclear.bat}:用于清理辅助文件。
\item \textbf{cdutlabreport.cls}:成都理工大学实验报告类文档,其中包括格式等全局设置,\textbf{除特殊情况请勿更改}
\end{itemize}
\subsection{文档结构}
打开文件CDUTLabReport.tex文件后,内为文档内容结构
\begin{minted}{LaTeX}
% !TeX encoding = UTF8
% !TEX TXS-program:compile = txs:///xelatex/[-shell-escape]
\documentclass{cdutlabreport}
%----------------------------------------title page setting-----------------
\course{\LaTeX 书写行为规范}
\academyname{利用\LaTeX 书写成都理工大学实验报告}
\profession{专业全称(有专业方向的用小括号标明)}
\studentName{您的姓名}
\studentId{您的学号}
\teacherName{您的授课或者指导老师}
\location{授课地点(如:6C403)}
\grade{由指导老师书写}
\startDate{二〇一九年十二月}
\lastDate{二〇二〇年二月}

\begin{document}
\include{cover/cover}
%----------------------------------------body--------------------------------
\include{body/update}
\include{body/section_1}
\include{body/section_2}
\include{body/section_3}
\include{body/section_4}
\include{body/section_5}
%----------------------------------------attachment--------------------------
\include{attachment/attachment}
\end{document}
\end{minted}

\verb|\doucumentclass[...]\{...}|称为设置文档类型,本模板使用成都理工大学的cdutlabreport类型。

在\verb|\doucumentclass|与\verb|\begin{document}|中间称为\textbf{导言区},学生需要在这里依次填写实验课程、实验名称、专业名称、学生姓名、学生学号、指导教师、实验地点和实验成绩,代替中括号内的内容即可。

文档内容使用\verb|\include{file}|逐个添加,便于整体管理的同时也使主文档美观简介。cover为文档封面与目录,\textbf{除特殊情况请勿更改},body内为文档各章节,attachment为学生实验心得。

文档实验报告内容从\verb|\begin{document}|开始,结束于\verb|\end{document}|。

由$\% !TeX$开头的两行为texstudio编辑器特有的魔法注释,意为使用UTF-8编码,编译引擎使用xelatex,由于本文代码高亮采用\verb|minted|包,所以添加了\verb|-shell-escape|,\textbf{若读者电脑中没有python环境,则无法正常编辑本文档,但不影响使用}。
\include{body/section_2}
\include{body/section_3}
\include{body/section_4}
\include{body/section_5}
%----------------------------------------attachment--------------------------
\include{attachment/attachment}
\end{document}
\end{minted}

\verb|\doucumentclass[...]\{...}|称为设置文档类型,本模板使用成都理工大学的cdutlabreport类型。

在\verb|\doucumentclass|与\verb|\begin{document}|中间称为\textbf{导言区},学生需要在这里依次填写实验课程、实验名称、专业名称、学生姓名、学生学号、指导教师、实验地点和实验成绩,代替中括号内的内容即可。

文档内容使用\verb|\include{file}|逐个添加,便于整体管理的同时也使主文档美观简介。cover为文档封面与目录,\textbf{除特殊情况请勿更改},body内为文档各章节,attachment为学生实验心得。

文档实验报告内容从\verb|\begin{document}|开始,结束于\verb|\end{document}|。

由$\% !TeX$开头的两行为texstudio编辑器特有的魔法注释,意为使用UTF-8编码,编译引擎使用xelatex,由于本文代码高亮采用\verb|minted|包,所以添加了\verb|-shell-escape|,\textbf{若读者电脑中没有python环境,则无法正常编辑本文档,但不影响使用}。
\include{body/section_2}
\include{body/section_3}
\include{body/section_4}
\include{body/section_5}
%----------------------------------------attachment--------------------------
\include{attachment/attachment}
\end{document}
\end{minted}

\verb|\doucumentclass[...]\{...}|称为设置文档类型,本模板使用成都理工大学的cdutlabreport类型。

在\verb|\doucumentclass|与\verb|\begin{document}|中间称为\textbf{导言区},学生需要在这里依次填写实验课程、实验名称、专业名称、学生姓名、学生学号、指导教师、实验地点和实验成绩,代替中括号内的内容即可。

文档内容使用\verb|\include{file}|逐个添加,便于整体管理的同时也使主文档美观简介。cover为文档封面与目录,\textbf{除特殊情况请勿更改},body内为文档各章节,attachment为学生实验心得。

文档实验报告内容从\verb|\begin{document}|开始,结束于\verb|\end{document}|。

由$\% !TeX$开头的两行为texstudio编辑器特有的魔法注释,意为使用UTF-8编码,编译引擎使用xelatex,由于本文代码高亮采用\verb|minted|包,所以添加了\verb|-shell-escape|,\textbf{若读者电脑中没有python环境,则无法正常编辑本文档,但不影响使用}。
\include{body/section_2}
\include{body/section_3}
\include{body/section_4}
\include{body/section_5}
%----------------------------------------attachment--------------------------
\include{attachment/attachment}
\end{document}
\end{minted}

\verb|\doucumentclass[...]\{...}|称为设置文档类型,本模板使用成都理工大学的cdutlabreport类型。

在\verb|\doucumentclass|与\verb|\begin{document}|中间称为\textbf{导言区},学生需要在这里依次填写实验课程、实验名称、专业名称、学生姓名、学生学号、指导教师、实验地点和实验成绩,代替中括号内的内容即可。

文档内容使用\verb|\include{file}|逐个添加,便于整体管理的同时也使主文档美观简介。cover为文档封面与目录,\textbf{除特殊情况请勿更改},body内为文档各章节,attachment为学生实验心得。

文档实验报告内容从\verb|\begin{document}|开始,结束于\verb|\end{document}|。

由$\% !TeX$开头的两行为texstudio编辑器特有的魔法注释,意为使用UTF-8编码,编译引擎使用xelatex,由于本文代码高亮采用\verb|minted|包,所以添加了\verb|-shell-escape|,\textbf{若读者电脑中没有python环境,则无法正常编辑本文档,但不影响使用}。